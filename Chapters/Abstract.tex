\kurzfassung{In den letzten Jahren ist der Schutz der Privatsphäre zu einem immer wichtigeren Thema in der Gesellschaft geworden. Mit dem Aufkommen des Internets, dem großen Boom der Blockchain-Technologie und der Verbreitung persönlicher Daten war die Notwendigkeit, unsere Informationen zu schützen, noch nie so groß. Eine Möglichkeit, die Privatsphäre zu schützen, ist die Verwendung von Zero-Knowledge-Proof. Der Zero-Knowledge-Beweis ist eine mathematische Methode, mit der eine Partei (der Beweisführer) einer anderen Partei (dem Verifizierer) beweisen kann, dass sie eine bestimmte Information kennt, ohne andere Informationen preiszugeben. 

Diese Arbeit zielt darauf ab, das Eigentum an einem Blockchain-basierten nicht-fungiblen Token (eine eindeutige digitale Repräsentation von materiellen oder immateriellen Objekten, in der Regel \ac{NFT}s) zu beweisen, ohne die Identität des Eigentümers preiszugeben. Das Ethereum-Netzwerk, die am weitesten verbreitete Blockchain-Technologie auf dem Markt, wird das Zielsystem sein. Die Arbeit befasst sich mit dem Problem, dass Menschen die Anonymität ihrer digitalen Vermögenswerte wahren müssen. Gleichzeitig können sie mehrere Adressen (digitale Geldbörsen) haben, von denen sie nicht wollen, dass öffentlich bekannt wird, dass sie die Besitzer sind.

Die Schritte zur Lösung dieser Probleme waren theoretische Forschung, Recherche und Auswahl aktueller Rahmenwerke oder Bibliotheken, Entwurf der Lösungsarchitektur, Implementierung und Test. Der erste Teil bestand in der Literaturrecherche und Lektüre. Der zweite Teil bestand darin, das Internet nach dem aktuellen Stand der Technik auf dem Gebiet des Nullwissens zu durchforsten. Außerdem wurde festgestellt, welche Bibliotheken und Frameworks zum Zeitpunkt der Recherche existierten. Nach der Auswahl der Frameworks und Bibliotheken war es im dritten Teil an der Zeit, eine Anwendungsarchitektur zur Lösung des Problems vorzuschlagen. Im vierten Teil wurde die agile Methodik eingesetzt, um mehrere Iterationen des Produkts zu erstellen und es mit den Benutzern zu testen. Schließlich haben wir Usability-Tests und Benchmarks durchgeführt, um die Leistung der Anwendung zu ermitteln.

Eine Schlussfolgerung nach all diesen Schritten ist, dass der aktuelle Stand der Technik von Zero-Knowledge praktische Anwendungsfälle in der realen Welt wie die Generierung des Eigentumsnachweises einer Blockchain-\ac{NFT} ohne Preisgabe der Identität des Eigentümers umsetzen kann. Es gibt jedoch Einschränkungen; angesichts der in dieser Arbeit verwendeten Strategie können wir diese nur statisch und nicht dynamisch nachweisen. Andere Implementierungen können jedoch dynamisch sein, ohne einen Blockchain-Snapshot zu verwenden.
}
\schlagworte{zk-SNARKs, Zero-Knowledge Proofs, Blockchain, Ethereum}
\outline{
In recent years, privacy has become an increasingly important issue in society. With the advent of the Internet, the great boom of blockchain technology and the proliferation of personal data, the need to protect our information has never been greater. One way to achieve privacy is through the use of zero-knowledge proof. Zero-knowledge proof is a mathematical method by which one party (the prover) can prove to another party (the verifier) that they know a particular piece of information without revealing any other information. 

The thesis aims to prove ownership of a blockchain-based non-fungible token (a unique digital representation of tangible or intangible objects, usually \ac{NFT}s) without compromising the owner's identity. The Ethereum network, the most prevalent blockchain technology in the market, will be the targeted system. The thesis addresses the problem of people maintaining the anonymity of their digital assets. At the same time, they may have multiple addresses (digital wallets), of which they do not want to be public knowledge that they are the owners.

The steps to resolve these issues were theoretical research, current frameworks or libraries research and selection, architecture design of the solution, implementation and testing. The first part was researching for a bibliography and reading. The second was scouting the Internet for the current state of the art in the field of zero knowledge. Moreover, seeing which libraries and frameworks exist at the current date of the research. In the third part, once the frameworks and libraries were selected, it was time to propose an application architecture to solve the problem. The fourth part used the Agile methodology to create multiple iterations of the product and test it with users. Finally, we used usability testing and benchmarks to see the application's performance of the product.

A conclusion after all these steps is that the current state of the art of zero-knowledge can implement practical use cases in the real world as generating the proof of ownership of a blockchain \ac{NFT} without compromising the owner's identity. However, it has limitations; given the strategy used in this thesis, we can only prove these in static ways and not dynamically. However, other implementations can be dynamic without using a blockchain snapshot.
}
\keywords{zk-SNARKs, Zero-Knowledge Proofs, Blockchain, Ethereum}