\chapter{Discussion}
In this study, we have presented an event management system that leverages zero-knowledge proofs to protect the privacy of attendees. The system combines a frontend, backend, and an Ethereum smart contract for efficient and secure operation. Through the use of zk-SNARKs, participants' identities remain confidential while still allowing for event organisers to manage and verify attendance.
Throughout the development and testing process, we encountered various challenges and limitations, which provided valuable insights into the current state of zero-knowledge technology. In this section, we will discuss some of the key observations and implications of our findings.

\section{Addressing Research Questions}
In the course of this study, we have addressed the research questions and provided insights into the practical implementation of zero-knowledge proofs for privacy-preserving event management. In this section, we will recap how each research question was answered:

\begin{enumerate}
\item \textbf{Is there a way to prove ownership without giving away personal information?}\
Yes, zero-knowledge proofs, such as zk-SNARKs, enable proving ownership without revealing personal information. This technology has been applied to our event management system to ensure participants' privacy while verifying attendance.
\item \textbf{What is Zero Knowledge Proof?}\\
A zero-knowledge proof is a cryptographic technique that allows one party to prove to another that they possess certain knowledge without revealing the underlying information. This concept was introduced in Chapter 2, where we discussed the principles of zero-knowledge proofs and their importance in preserving privacy.

\item \textbf{How does it work?}\\
Zero-knowledge proofs work by transforming statements about the secret information into equivalent statements that do not reveal the secret. In Chapter 2, we discussed zk-SNARKs, a specific type of zero-knowledge proof, and their application in our event management system.

\item \textbf{What are the existing implementations of Zero Knowledge Proofs in the context of blockchain technology?}\\
In Chapter 2, we examined existing implementations of zero-knowledge proofs in blockchain technology, such as Zcash and Aztec Protocol. We also analyzed their applications and limitations in the context of privacy-preserving systems.

\item \textbf{What are the limitations and challenges of using Zero-Knowledge Proofs for event management?}\\
Some limitations and challenges include scalability, ease of use, integration with existing systems, and legal and regulatory compliance. These issues were discussed in the Discussion chapter, where we explored their implications and potential solutions.

\item \textbf{How can Zero Knowledge Proofs be integrated into existing event management platforms and ecosystems?}\\
In Chapter 3, we described the design and implementation of our event management system, which integrated zero-knowledge proofs into its core functionalities. This served as a practical example of how zero-knowledge proofs can be incorporated into existing event management platforms and ecosystems to enhance privacy protection.

\item \textbf{What potential security risks are associated with Zero Knowledge Proofs, and how can they be mitigated?}\\
Security risks associated with zero-knowledge proofs include potential vulnerabilities in the underlying cryptographic algorithms, implementation errors, and attacks on the anonymity set. In the course of this study, we discussed possible mitigation strategies, such as using well-vetted cryptographic libraries, rigorous testing, and regular security audits.

\end{enumerate}

The hypothesis formulated at the beginning of this thesis was: ``Can the current state-of-the-art of Zero Knowledge Proof be applied to create a practical solution for privacy-preserving event management?'' Based on the findings and discussions presented throughout this study, we can conclude that the current state-of-the-art of zero-knowledge proofs can indeed be applied to create practical solutions for privacy-preserving event management, as demonstrated by our event management system.

\section{Scalability}
One of the primary concerns with the system is its scalability, particularly when dealing with a large number of event attendees. As the number of participants increases, the time required to generate zk-SNARK proofs and the complexity of the Merkle tree used to store the commitments may also grow. This could potentially lead to performance bottlenecks, which may affect the overall user experience. Further research is necessary to optimise the performance and scalability of the system to accommodate larger events.

\section{Ease of Use}
While the presented system is functional, it may not be as user-friendly as traditional event management systems. The use of cryptographic techniques, such as zk-SNARKs, may introduce complexities unfamiliar to the average user. It is essential to develop intuitive user interfaces and provide educational resources to ensure users can effectively use the system without compromising their privacy.

\section{Integration with Existing Systems}
Our event management system is designed as a standalone application. However, for widespread adoption, it may be necessary to integrate it with existing event management platforms or social media networks. This would require additional research and development to ensure compatibility and a seamless user experience.

\section{Legal and Regulatory Compliance}
As with any technology dealing with personal data and privacy, it is crucial to consider legal and regulatory compliance. Different jurisdictions may have varying regulations surrounding data protection, which may impact the implementation and use of our event management system. It is essential to ensure the system complies with all relevant laws and regulations to avoid potential legal issues.

\section{Future Research and Development}
The field of zero-knowledge proofs is rapidly evolving, with new libraries, techniques, and optimisations being introduced regularly. As this technology advances, it presents opportunities to enhance the event management system further. Future research and development efforts may focus on improving scalability, user experience, and integration with other platforms.

In conclusion, the event management system presented in this study demonstrates the potential of zero-knowledge proofs in protecting the privacy of event attendees. While there are limitations and challenges to overcome, the rapid growth and development in this field suggest that zero-knowledge technology will continue to play a significant role in ensuring privacy and security in various applications.

