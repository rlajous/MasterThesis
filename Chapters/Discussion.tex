\chapter{Discussion}
In this study, we have presented an event management system that leverages zero-knowledge proofs to protect the privacy of attendees. The system combines a frontend, backend, and an Ethereum smart contract for efficient and secure operation. Through the use of zk-SNARKs, participants' identities remain confidential while still allowing for event organisers to manage and verify attendance.
Throughout the development and testing process, we encountered various challenges and limitations, which provided valuable insights into the current state of zero-knowledge technology. In this section, we will discuss some of the key observations and implications of our findings.

\section{Scalability}
One of the primary concerns with the system is its scalability, particularly when dealing with a large number of event attendees. As the number of participants increases, the time required to generate zk-SNARK proofs and the complexity of the Merkle tree used to store the commitments may also grow. This could potentially lead to performance bottlenecks, which may affect the overall user experience. Further research is necessary to optimise the performance and scalability of the system to accommodate larger events.

\section{Ease of Use}
While the presented system is functional, it may not be as user-friendly as traditional event management systems. The use of cryptographic techniques, such as zk-SNARKs, may introduce complexities unfamiliar to the average user. It is essential to develop intuitive user interfaces and provide educational resources to ensure users can effectively use the system without compromising their privacy.

\section{Integration with Existing Systems}
Our event management system is designed as a standalone application. However, for widespread adoption, it may be necessary to integrate it with existing event management platforms or social media networks. This would require additional research and development to ensure compatibility and a seamless user experience.

\section{Legal and Regulatory Compliance}
As with any technology dealing with personal data and privacy, it is crucial to consider legal and regulatory compliance. Different jurisdictions may have varying regulations surrounding data protection, which may impact the implementation and use of our event management system. It is essential to ensure the system complies with all relevant laws and regulations to avoid potential legal issues.

\section{Future Research and Development}
The field of zero-knowledge proofs is rapidly evolving, with new libraries, techniques, and optimisations being introduced regularly. As this technology advances, it presents opportunities to enhance the event management system further. Future research and development efforts may focus on improving scalability, user experience, and integration with other platforms.

In conclusion, the event management system presented in this study demonstrates the potential of zero-knowledge proofs in protecting the privacy of event attendees. While there are limitations and challenges to overcome, the rapid growth and development in this field suggest that zero-knowledge technology will continue to play a significant role in ensuring privacy and security in various applications.

