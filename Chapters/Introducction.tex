\chapter{Introduction}
The internet and digital technologies have revolutionised how we live and work. However, it has also created new security, encryption, and anonymity challenges. In this era of global connectivity, data is increasingly vulnerable to theft, leakage, and hacking. All this has led to a heightened need for security and encryption and greater awareness of the importance of anonymity.

Several standard and current problems with security, encryption, and anonymity exist. One of the most significant challenges is the increasing sophistication of cyber-attacks. Hackers are constantly finding new ways to penetrate systems and steal data, making it difficult for businesses and individuals to keep their data safe.

Another problem is the weak security of many internet-connected devices, including everything from laptops and smartphones to home routers and security cameras. Many of these devices are easy to hack into, giving cyber-criminals access to personal and sensitive data. Additionally, outdated security methods, such as password-based authentication and lack of data encryption, continue to be prevalent in many organisations, leaving them vulnerable to attack.

This thesis addresses privacy issues surrounding \ac{NFT} proof of ownership in Ethereum.

\section{Motivation}
Proving ownership of an \ac{NFT} enables various use cases regarding token-gated benefits, such as access to exclusive content, discounts, and participation in events such as conferences, meetup, concerts and so on. However, current methods for proving ownership may inadvertently expose personal information and transaction history, which raises privacy concerns for users. Implementing a more privacy-preserving approach, such as Zero Knowledge Proofs, could enhance the attractiveness and security of \ac{NFT}-based platforms while maintaining the integrity of the blockchain and providing users with increased control over their personal information.

\subsection{The problem}
When a person makes a transaction on the blockchain, that information remains permanently, creating an immutable record. By proving their ownership of an asset, individuals may inadvertently expose sensitive information that can reveal their entire transaction history on the blockchain. This lack of privacy contrasts with other blockchains, such as Bitcoin, which allows users to generate specific addresses for each transaction, thus enhancing their anonymity. With the increasing popularity and adoption of \ac{NFT}s in various industries, the need for privacy-preserving methods of proving ownership has become critical. The challenge lies in developing an approach that maintains the benefits of transparency and decentralisation while protecting users' personal information and ensuring their anonymity.

\section{Main Goals}
This research aims to develop a platform that enables users to access token-gated benefits through Zero Knowledge Proofs, ensuring privacy and anonymity in proving \ac{NFT} ownership. The following objectives will be pursued:

\begin{enumerate}
\item \textbf{Investigate} the current state-of-the-art Zero Knowledge Proofs and their applications in blockchain technology.
\item \textbf{Identify} the challenges and limitations of using Zero-Knowledge Proofs for \ac{NFT} ownership verification.
\item \textbf{Propose} a novel architecture for integrating Zero Knowledge Proofs into \ac{NFT} platforms and ecosystems.
\item \textbf{Develop} a minimum viable product (MVP) that generates proof of \ac{NFT} ownership without revealing personal information. The MVP will involve a user sending a proof request to a verifier (e.g., a bouncer), who will then compare the user's response with the expected proof, ensuring a match without disclosing sensitive information.
\end{enumerate}

By achieving these goals, this research aims to contribute to the ongoing development of privacy-preserving methods in blockchain technology, specifically in \ac{NFT} ownership verification.

\section{Research Questions}
Before formulating the hypothesis, the following research questions were explored:

\begin{enumerate}
\item Is there a way to prove ownership without personal information?
\item What is Zero Knowledge Proof?
\item How does it work?
\item Are implementations of Zero Knowledge Proofs in the context of blockchain technology?
\item What are the limitations and challenges of using Zero-Knowledge Proofs for \ac{NFT} ownership verification?
\item How can Zero Knowledge Proofs be integrated into existing \ac{NFT} platforms and ecosystems?
\item What potential security risks are associated with Zero Knowledge Proofs, and how can they be mitigated?
\end{enumerate}

This paper will answer these questions and provide insights into the practical implementation of Zero-Knowledge Proofs for privacy-preserving \ac{NFT} ownership verification. The thesis hypothesis is formulated as follows: ``Can the current state-of-the-art of Zero Knowledge Proof be applied to create a practical solution for privacy-preserving \ac{NFT} ownership verification?''.